%uncomment for thinner margins:
%\usepackage[top = 2.5cm, bottom = 2.5cm, left = 2.5cm, right = 2.5cm]{geometry}
\usepackage[T1]{fontenc}
\usepackage[utf8]{inputenc}
\usepackage[style=alphabetic]{biblatex}
\usepackage{graphicx,enumitem,setspace,float,fancyhdr,subfiles}
\usepackage{amsmath,mathtools,amssymb,amsthm,cancel,mathdots,tikz-cd,xfrac,xcolor,amsfonts}
\usepackage[all]{xy}
\usepackage[mathscr]{euscript}
\usepackage{hyperref}

%scale tikzcd
\tikzcdset{scale cd/.style={every label/.append style={scale=#1},
    cells={nodes={scale=#1}}}}
% \setlength{\parindent}{0in}
\makeatletter
\DeclareMathOperator{\oeq}{\mathbin{\mathpalette\make@circled=}}%get your circled symbols here!
\DeclareMathOperator{\oneq}{\mathbin{\mathpalette\make@circled\neq}}
\newcommand{\make@circled}[2]{%
  \ooalign{$\m@th#1\smallbigcirc{#1}$\cr\hidewidth$\m@th#1#2$\hidewidth\cr}%
}
\newcommand{\smallbigcirc}[1]{%
  \vcenter{\hbox{\scalebox{0.77778}{$\m@th#1\bigcirc$}}}%
}
\makeatother
\newcommand{\bA}{\mathbb{A}}
\newcommand{\bC}{\mathbb{C}}
\newcommand{\bF}{\mathbb{F}}
\newcommand{\bN}{\mathbb{N}}
\newcommand{\bP}{\mathbb{P}}
\newcommand{\bQ}{\mathbb{Q}}
\newcommand{\bR}{\mathbb{R}}
\newcommand{\bZ}{\mathbb{Z}}
\newcommand{\lt}{\left}
\newcommand{\rt}{\right}
\newcommand{\rgl}{\rangle}
\newcommand{\lgl}{\langle}
%change the style of arrows
\newcommand{\ra}{\rightarrow}
\newcommand{\lra}{\leftrightarrow}
\newcommand{\xra}{\xrightarrow}
\newcommand{\Ra}{\Rightarrow}
\newcommand{\Lra}{\Leftrightarrow}
\newcommand{\e}{\varepsilon}
%\det, \dim, \ker, \lcm, \gcd, all trig functions, already built-in
\DeclareMathOperator{\coker}{\operatorname{coker}}
\DeclareMathOperator{\id}{\operatorname{id}}
\DeclareMathOperator{\obj}{\operatorname{obj}}
\DeclareMathOperator{\im}{\operatorname{im}}
\DeclareMathOperator{\tr}{\operatorname{tr}}
\DeclareMathOperator{\End}{\operatorname{End}}
\DeclareMathOperator{\Hom}{\operatorname{Hom}}
\DeclareMathOperator{\Ext}{\operatorname{Ext}}
\DeclareMathOperator{\Tor}{\operatorname{Tor}}
\DeclareMathOperator{\Span}{\operatorname{Span}}
\newcommand{\mychi}{{\raise 2pt\hbox{$\chi$}}}
\definecolor{red2}{RGB}{160, 50, 50}
\definecolor{green2}{RGB}{50, 160, 50}
\definecolor{blue2}{RGB}{50, 50, 160}
\definecolor{magenta2}{RGB}{160, 48, 160}
\definecolor{cyan2}{RGB}{48, 160, 160}
\definecolor{yellow2}{RGB}{160, 160, 48}

%AMS-style theorem environments:

\theoremstyle{plain}
\newtheorem{theorem}{Theorem}
\newtheorem{proposition}[theorem]{Proposition}
\newtheorem{preposition}[theorem]{Preposition}
\newtheorem{lemma}[theorem]{Lemma}
\newtheorem{corollary}[theorem]{Corollary}
%solution for custom numbering in theorems and lemmata:
\newtheorem{innercustomgeneric}{\customgenericname}
\providecommand{\customgenericname}{}
\newcommand{\newcustomtheorem}[2]{%
  \newenvironment{#1}[1]
  {%
   \renewcommand\customgenericname{#2}%
   \renewcommand\theinnercustomgeneric{##1}%
   \innercustomgeneric
  }
  {\endinnercustomgeneric}
}
\newcustomtheorem{customthm}{Theorem}
\newcustomtheorem{customlemma}{Lemma}
\newcustomtheorem{customcor}{Corollary}
\newcustomtheorem{customprop}{Proposition}
%%%%%

\theoremstyle{definition}
\newtheorem{definition}{Definition}[section]
\newtheorem{example}[theorem]{Example}
\newtheorem{exercise}{Exercise}

\theoremstyle{remark}
\newtheorem{remark}[theorem]{Remark}

%%%%%%%%%%%%%%%%%%%%%%%%%%%%%%%%%%%%%%%%%%%%%%%%
% 3. Header (and Footer)
%%%%%%%%%%%%%%%%%%%%%%%%%%%%%%%%%%%%%%%%%%%%%%%%
% To make our document nice we want a header and number the pages in the footer.
\pagestyle{fancy} % With this command we can customize the header style.
\fancyhf{} % This makes sure we do not have other information in our header or footer.
\lhead{\footnotesize {}}% \lhead puts text in the top left corner. \footnotesize sets our font to a smaller size.
%\rhead works just like \lhead (you can also use \chead)
\chead{\footnotesize {}}
\rhead{\footnotesize {}} %<---- Fill in your lastnames.
% Similar commands work for the footer (\lfoot, \cfoot and \rfoot).
% We want to put our page number in the center.
\cfoot{\footnotesize \thepage}