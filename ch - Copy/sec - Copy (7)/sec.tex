\section{zin3724}
Let $G$ be the Lie group $S^1:=\left\{ \bR\mod2\pi,+ \right\}$. Then each $\theta$ induces a map $S^1\xrightarrow{+\theta} S^1$ as $G$-spaces, given by
\[+\theta(\omega)=\omega+\theta\mod{2\pi}\]
for all $\omega\in S^1$. Here, $G$ acts by left multiplication in both cases. The equalizer of $\{+\theta\mid\theta\in[0,2\pi)\}$ in the category of $G$-spaces is $S^1\xra{\id_{S^1}}$, but each nonzero $+\theta$ has no fixed points, so in $\textbf{Sets}$, the equalizer is $\emptyset$, which isn't the underlying set of the $G$-set $S^1$, so we have a counterexample to the claim that the forgetful functor $U\colon\textbf{B}G\ra\textbf{Sets}$ preserves limits.