\section{zin3724}
This is almost exactly the same as the proof of Proposition 1.6.1 on pp. 46-47. Note that $D_B$ is a forgetful-like functor from $\mathbf C/B$ to $\mathbf C$ that takes each object in $\mathbf C/B$ to its domain.

Suppose $\phi$ is a natural trasformation in $\widehat{\mathbf{C}}(R\times P,Q)$, then we define for each $B\in\obj(\mathbf C)$ the component $\phi'_B$ such that for each $u\in R(B)$, the image $\phi'_B(u)\in\widehat{\mathbf{C}/B}(P_B,Q_B)$ has its component at each $\lt(C\xra{c}B\rt)\in\obj(\mathbf C/B)$ defined as
\[(\phi'_B(u))_c:P(C)\to Q(C):y\mapsto\phi_C(u\cdot c,y)=\phi_C(R(c)(u),y)\]

To see that the components $(\phi'_B(u))_c$ indeed form a natural trasformation in $\widehat{\mathbf{C}/B}(P_B,Q_B)$, suppose $k:G\to H$ is a morphism from $G\xra{g}B$ to $H\xra{h}B$ in $\mathbf{C}/B$, then for any $u\in R(B)$ and $y\in P(H)$
\[(\phi'_G(u))_g\circ P(k)(y)=\phi_G(u\cdot g,P(k)(y))=\phi_G(R(g)(u),P(k)(y))\]
on the other hand, by the naturality of $\phi$ and the functoriality of $R$,
\[Q(k)\circ(\phi'_H(u))_h(y)=Q(k)\circ\phi_H(u\cdot h,y)=Q(k)\circ\phi_H(R(h)(u),y)\]
\[=\phi_G(R(k)\circ R(h)(u),P(k)(y))=\phi_G(R(h\circ k)(u),P(k)(y))\]
\[=\phi_G(R(g)(u),P(k)(y))=(\phi'_G(u))_g\circ P(k)(y)\]

Now, we define the evaluation map $ev$ from $\widehat{\mathbf{C}/(-)}(P_{(-)},Q_{(-)})\times P$ to $Q$, by defining for each $B\in\obj(\mathbf{C})$, its component as
\[(ev)_B:\widehat{\mathbf{C}/B}(P_B,Q_B)\times P(B)\to Q(B):(\alpha,w)\mapsto\alpha_{1_B}(w)\]
so as an example, for $(u,w)\in R(B)\times P(B)$, we have
\[(ev)_B(\phi'_B(u),w)=(\phi'_B(u))_{1_B}(w)=\phi_B(u\cdot1_B,w)=\phi_B(R(1_B)(u),w)\]
\[=\phi_B(1_{R(B)}(u),w)=\phi_B(u,w)\]
hence each $\phi\in\widehat{\mathbf{C}}(R\times P,Q)$ factors through $ev$. Now, it follows from the way we costructed each $\phi'_B$ and the constraint that $\phi'$ has to be a natural trasformation from $R$ to $\widehat{\mathbf{C}/(-)}(P_{(-)},Q_{(-)})$, that $\phi$ factors uniquely through $ev$, so $\widehat{\mathbf{C}/(-)}(P_{(-)},Q_{(-)})$ can be regarded as the exponential $Q^P$ in $\widehat{\mathbf{C}}$.