\section{zin3724}
\emph{dedicated to sharp2229, previously known as Dir (see the cursed music theory lecture "Abstracted Systems of Music", on YouTube), Darkrifts}

This is almost exactly the same as the proof of Proposition 1.6.1 on pp. 46-47. Note that $D_B$ is a forgetful-like functor from $\mathbf C/B$ to $\mathbf C$ that takes each object in $\mathbf C/B$ to its domain.

Suppose $\phi$ is a natural trasformation in $\widehat{\mathbf{C}}(R\times P,Q)$, then we define for each $B\in\obj(\mathbf C)$ the component $\phi'_B$ such that for each $u\in R(B)$, the image $\phi'_B(u)\in\widehat{\mathbf{C}/B}(P_B,Q_B)$ has its component at each $\lt(C\xra{c}B\rt)\in\obj(\mathbf C/B)$ defined as
\[(\phi'_B(u))_c:P(C)\to Q(C):y\mapsto\phi_C(u\cdot c,y)=\phi_C(R(c)(u),y)\]

To see that the components $(\phi'_B(u))_c$ indeed form a natural trasformation in $\widehat{\mathbf{C}/B}(P_B,Q_B)$, suppose $k:G\to H$ is a morphism from $G\xra{g}B$ to $H\xra{h}B$ in $\mathbf{C}/B$, then for any $u\in R(B)$ and $y\in P(H)$
\[(\phi'_G(u))_g\circ P(k)(y)=\phi_G(u\cdot g,P(k)(y))=\phi_G(R(g)(u),P(k)(y))\]
on the other hand, by the naturality of $\phi$ and the functoriality of $R$,
\[Q(k)\circ(\phi'_H(u))_h(y)=Q(k)\circ\phi_H(u\cdot h,y)=Q(k)\circ\phi_H(R(h)(u),y)\]
\[=\phi_G(R(k)\circ R(h)(u),P(k)(y))=\phi_G(R(h\circ k)(u),P(k)(y))\]
\[=\phi_G(R(g)(u),P(k)(y))=(\phi'_G(u))_g\circ P(k)(y)\]

For brevity, we now write $Q^P$ to be $\widehat{\mathbf{C}/(-)}(P_{(-)},Q_{(-)}):\mathbf{C}^{op}\to\mathbf{Sets}$, regardless of whether it is such an exponential object, or not. $Q^P$ maps each morphism $(B\xra{f^{op}}A)$ in $\mathbf{C}^{op}$ by mapping $\beta\in Q^P(B)$ to $\alpha\in Q^P(A)$ componentwise via
\[\alpha_{x^{op}}\coloneqq\beta_{(f\circ x)^{op}}\]
for all $(A\xra{x^{op}}X)\in(\mathbf{C}/A)^{op}$.

To see where that came from, since each $(X\xra{x}A)\in\obj(\mathbf{C}/A)$ can be mapped to $X\xra{f\circ x}B\in\obj(\mathbf{C}/B)$ such that they have the same underlying domain in $\mathbf{C}$,
% it follows that $D_A^{op}(x^{op})=D_B^{op}((f\circ x)^{op})$,
and since each morphism $Y\xra{g^{op}}X$ such that 
\[\xymatrix{
    X & Y \ar@{->}[l]^{g^{op}} \\
    A \ar@{->}[ru]_{y^{op}} \ar@{->}[u]^{x^{op}} & 
    }\]
commutes is also a morphism such that
\[\xymatrix{
    X & Y \ar@{->}[l]^{g^{op}} \\
    A \ar@{->}[ru]_{y^{op}} \ar@{->}[u]^{x^{op}} &  \\
    B \ar@{->}[u]^{f^{op}} & 
    }\]
commutes, it follows that each morphism in $(\mathbf{C}/A)^{op}$ is identically a morphism in $(\mathbf{C}/B)^{op}$. Now, we have a functor $(\mathbf{C}/f)^{op}:(\mathbf{C}/A)^{op}\to(\mathbf{C}/B)^{op}$. Clearly, $Q^P(f):Q^P(B)\to Q^P(A)$ is the function $\beta\mapsto\beta((\mathbf{C}/f)^{op})$, which takes $\beta$ and precomposes it horizontally with the functor $(\mathbf{C}/f)^{op}$ ("pre-whiskering" in Emily Riehl's terminology?). We can now conclude from the assumption $\beta\in Q^P(B)$, that $\alpha$ is indeed a natural transformation in $Q^P(A)$.

To see that $Q^P$ is a functor in $\mathbf{Sets}^{\mathbf{C}^{op}}$, note that for all $B\in\obj(\mathbf{C})$, $(\mathbf{C}/\id_B)^{op}$ is the identity functor on $(\mathbf{C}/B)^{op}$, so $Q^P(\id_B)=\id_{Q^P(B)}$. Now, suppose that $(A\xra{f}B),(B\xra{g}C)\in\mathbf{C}$, then $(\mathbf{C}/(g\circ f))^{op}=(\mathbf{C}/f)^{op}(\mathbf{C}/g)^{op}$, so $Q^P((g\circ f)^{op})=Q^P(f^{op})\circ Q^P(g^{op})$, all because of how precomposing functors work.

To see that $\phi':R\to Q^P$ is natural, suppose $u\in R(B)$, and $(A\xra{f}B)\in\mathbf{C}$. Then $\phi'_A\circ R(f)(u)$ is the natural transformation such that its component at $X\xra{x}A$ is
\[\phi_X(R(x)(R(f)(u)),-)=\phi_X(R(f\circ x)(u),-)\]
on the other hand, $Q^P(f)(\phi'_B(u))=\phi'_B(u)((\mathbf{C}/f)^{op})$ is the natural transformation such that its component at $x$ is
\[(\phi'_B(u))_{f\circ x}=\phi_X(R(f\circ x)(u),-)\]
so it follows that $Q^P(f)\circ\phi'_B=\phi'_A\circ Q^P(f)$ as required.

Now, we define the evaluation map $ev$ from $\widehat{\mathbf{C}/(-)}(P_{(-)},Q_{(-)})\times P$ to $Q$, by defining for each $B\in\obj(\mathbf{C})$, its component as
\[(ev)_B:\widehat{\mathbf{C}/B}(P_B,Q_B)\times P(B)\to Q(B):(\beta,w)\mapsto\beta_{1_B}(w)\]
so as an example, for $(u,w)\in R(B)\times P(B)$, we have
\[(ev)_B(\phi'_B(u),w)=(\phi'_B(u))_{1_B}(w)=\phi_B(u\cdot1_B,w)=\phi_B(R(1_B)(u),w)\]
\[=\phi_B(1_{R(B)}(u),w)=\phi_B(u,w)\]
hence each component of $\phi\in\widehat{\mathbf{C}}(R\times P,Q)$ factors through the corresponding component of $ev$.

To see that $ev$ is natural, for any $(A\xra{f}B)\in\mathbf{C}$ and $(\beta,w)\in Q^P(B)\times P(B)$, note that
\[ev_A\circ (Q^P(f)\times P(f))(\beta,w)=ev_A(\beta((\mathbf{C}/f)^{op}),P(f)(w))\]
\[=(\beta((\mathbf{C}/f)^{op}))_{1_A}(P(f)(w))=\beta_{f^{op}}(P(f)(w))\in Q(A)\]
and on the other hand,
\[Q(f)\circ ev_B(\beta,w)=Q(f)\beta_{1_B}(w)\]
moreover, it follows from the naturality of $\beta$ (as a natural transformation in $\widehat{\mathbf{C}/B}(P_B,Q_B)$) that $Q(f)\beta_{1_B}=\beta_{f^{op}}P(f)$, so we indeed have $ev_A\circ (Q^P(f)\times P(f))=Q(f)\circ ev_B$.

Now, it follows from the way we costructed each $\phi'_B$ and the constraint that $\phi'$ has to be a natural trasformation from $R$ to $\widehat{\mathbf{C}/(-)}(P_{(-)},Q_{(-)})$, that $\phi$ factors uniquely through $ev$, so $Q^P(-)=\widehat{\mathbf{C}/(-)}(P_{(-)},Q_{(-)})$ is indeed such an exponential in $\widehat{\mathbf{C}}$.